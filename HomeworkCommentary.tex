% Options for packages loaded elsewhere
\PassOptionsToPackage{unicode}{hyperref}
\PassOptionsToPackage{hyphens}{url}
%
\documentclass[
]{article}
\title{HomeworkCommentary}
\author{Murray Chapman}
\date{29/10/2021}

\usepackage{amsmath,amssymb}
\usepackage{lmodern}
\usepackage{iftex}
\ifPDFTeX
  \usepackage[T1]{fontenc}
  \usepackage[utf8]{inputenc}
  \usepackage{textcomp} % provide euro and other symbols
\else % if luatex or xetex
  \usepackage{unicode-math}
  \defaultfontfeatures{Scale=MatchLowercase}
  \defaultfontfeatures[\rmfamily]{Ligatures=TeX,Scale=1}
\fi
% Use upquote if available, for straight quotes in verbatim environments
\IfFileExists{upquote.sty}{\usepackage{upquote}}{}
\IfFileExists{microtype.sty}{% use microtype if available
  \usepackage[]{microtype}
  \UseMicrotypeSet[protrusion]{basicmath} % disable protrusion for tt fonts
}{}
\makeatletter
\@ifundefined{KOMAClassName}{% if non-KOMA class
  \IfFileExists{parskip.sty}{%
    \usepackage{parskip}
  }{% else
    \setlength{\parindent}{0pt}
    \setlength{\parskip}{6pt plus 2pt minus 1pt}}
}{% if KOMA class
  \KOMAoptions{parskip=half}}
\makeatother
\usepackage{xcolor}
\IfFileExists{xurl.sty}{\usepackage{xurl}}{} % add URL line breaks if available
\IfFileExists{bookmark.sty}{\usepackage{bookmark}}{\usepackage{hyperref}}
\hypersetup{
  pdftitle={HomeworkCommentary},
  pdfauthor={Murray Chapman},
  hidelinks,
  pdfcreator={LaTeX via pandoc}}
\urlstyle{same} % disable monospaced font for URLs
\usepackage[margin=1in]{geometry}
\usepackage{color}
\usepackage{fancyvrb}
\newcommand{\VerbBar}{|}
\newcommand{\VERB}{\Verb[commandchars=\\\{\}]}
\DefineVerbatimEnvironment{Highlighting}{Verbatim}{commandchars=\\\{\}}
% Add ',fontsize=\small' for more characters per line
\usepackage{framed}
\definecolor{shadecolor}{RGB}{248,248,248}
\newenvironment{Shaded}{\begin{snugshade}}{\end{snugshade}}
\newcommand{\AlertTok}[1]{\textcolor[rgb]{0.94,0.16,0.16}{#1}}
\newcommand{\AnnotationTok}[1]{\textcolor[rgb]{0.56,0.35,0.01}{\textbf{\textit{#1}}}}
\newcommand{\AttributeTok}[1]{\textcolor[rgb]{0.77,0.63,0.00}{#1}}
\newcommand{\BaseNTok}[1]{\textcolor[rgb]{0.00,0.00,0.81}{#1}}
\newcommand{\BuiltInTok}[1]{#1}
\newcommand{\CharTok}[1]{\textcolor[rgb]{0.31,0.60,0.02}{#1}}
\newcommand{\CommentTok}[1]{\textcolor[rgb]{0.56,0.35,0.01}{\textit{#1}}}
\newcommand{\CommentVarTok}[1]{\textcolor[rgb]{0.56,0.35,0.01}{\textbf{\textit{#1}}}}
\newcommand{\ConstantTok}[1]{\textcolor[rgb]{0.00,0.00,0.00}{#1}}
\newcommand{\ControlFlowTok}[1]{\textcolor[rgb]{0.13,0.29,0.53}{\textbf{#1}}}
\newcommand{\DataTypeTok}[1]{\textcolor[rgb]{0.13,0.29,0.53}{#1}}
\newcommand{\DecValTok}[1]{\textcolor[rgb]{0.00,0.00,0.81}{#1}}
\newcommand{\DocumentationTok}[1]{\textcolor[rgb]{0.56,0.35,0.01}{\textbf{\textit{#1}}}}
\newcommand{\ErrorTok}[1]{\textcolor[rgb]{0.64,0.00,0.00}{\textbf{#1}}}
\newcommand{\ExtensionTok}[1]{#1}
\newcommand{\FloatTok}[1]{\textcolor[rgb]{0.00,0.00,0.81}{#1}}
\newcommand{\FunctionTok}[1]{\textcolor[rgb]{0.00,0.00,0.00}{#1}}
\newcommand{\ImportTok}[1]{#1}
\newcommand{\InformationTok}[1]{\textcolor[rgb]{0.56,0.35,0.01}{\textbf{\textit{#1}}}}
\newcommand{\KeywordTok}[1]{\textcolor[rgb]{0.13,0.29,0.53}{\textbf{#1}}}
\newcommand{\NormalTok}[1]{#1}
\newcommand{\OperatorTok}[1]{\textcolor[rgb]{0.81,0.36,0.00}{\textbf{#1}}}
\newcommand{\OtherTok}[1]{\textcolor[rgb]{0.56,0.35,0.01}{#1}}
\newcommand{\PreprocessorTok}[1]{\textcolor[rgb]{0.56,0.35,0.01}{\textit{#1}}}
\newcommand{\RegionMarkerTok}[1]{#1}
\newcommand{\SpecialCharTok}[1]{\textcolor[rgb]{0.00,0.00,0.00}{#1}}
\newcommand{\SpecialStringTok}[1]{\textcolor[rgb]{0.31,0.60,0.02}{#1}}
\newcommand{\StringTok}[1]{\textcolor[rgb]{0.31,0.60,0.02}{#1}}
\newcommand{\VariableTok}[1]{\textcolor[rgb]{0.00,0.00,0.00}{#1}}
\newcommand{\VerbatimStringTok}[1]{\textcolor[rgb]{0.31,0.60,0.02}{#1}}
\newcommand{\WarningTok}[1]{\textcolor[rgb]{0.56,0.35,0.01}{\textbf{\textit{#1}}}}
\usepackage{graphicx}
\makeatletter
\def\maxwidth{\ifdim\Gin@nat@width>\linewidth\linewidth\else\Gin@nat@width\fi}
\def\maxheight{\ifdim\Gin@nat@height>\textheight\textheight\else\Gin@nat@height\fi}
\makeatother
% Scale images if necessary, so that they will not overflow the page
% margins by default, and it is still possible to overwrite the defaults
% using explicit options in \includegraphics[width, height, ...]{}
\setkeys{Gin}{width=\maxwidth,height=\maxheight,keepaspectratio}
% Set default figure placement to htbp
\makeatletter
\def\fps@figure{htbp}
\makeatother
\setlength{\emergencystretch}{3em} % prevent overfull lines
\providecommand{\tightlist}{%
  \setlength{\itemsep}{0pt}\setlength{\parskip}{0pt}}
\setcounter{secnumdepth}{-\maxdimen} % remove section numbering
\ifLuaTeX
  \usepackage{selnolig}  % disable illegal ligatures
\fi

\begin{document}
\maketitle

\hypertarget{week-5-homework}{%
\section{Week 5 Homework}\label{week-5-homework}}

By Murray Chapman

\hypertarget{introduction}{%
\subsection{Introduction}\label{introduction}}

The following code produces a world map colour coded according to
changes in inequality level change between the years 2010 and 2019
according to the UN Gender Inequality Index

\hypertarget{the-code}{%
\subsection{The code}\label{the-code}}

\begin{Shaded}
\begin{Highlighting}[]
\FunctionTok{library}\NormalTok{(here) }\CommentTok{\#Load all the necessary packages}
\FunctionTok{library}\NormalTok{(dplyr)}
\FunctionTok{library}\NormalTok{(janitor)}
\FunctionTok{library}\NormalTok{(stringr)}
\FunctionTok{library}\NormalTok{(sf)}
\FunctionTok{library}\NormalTok{(tidyverse)}
\FunctionTok{library}\NormalTok{(tmap)}
\FunctionTok{library}\NormalTok{(tmaptools)}
\FunctionTok{library}\NormalTok{(countrycode)}

\CommentTok{\#Read in the shapefile containing country names and geographical data}
\CommentTok{\#"Here" removes the need for me to specify the full path, as it will start from the project folder}

\NormalTok{CountryData }\OtherTok{\textless{}{-}} \FunctionTok{st\_read}\NormalTok{(}\StringTok{"World\_Countries\_(Generalized)"}\NormalTok{) }\SpecialCharTok{\%\textgreater{}\%}
  \FunctionTok{clean\_names}\NormalTok{() }\CommentTok{\#Neatens the column names, removing the capitalization}
\end{Highlighting}
\end{Shaded}

\begin{verbatim}
## Reading layer `World_Countries__Generalized_' from data source 
##   `/Users/murray/Documents/MResSDSV/GIS/Week4/GIS_Code/World_Countries_(Generalized)' 
##   using driver `ESRI Shapefile'
## Simple feature collection with 249 features and 7 fields
## Geometry type: MULTIPOLYGON
## Dimension:     XY
## Bounding box:  xmin: -180 ymin: -89 xmax: 180 ymax: 83.6236
## Geodetic CRS:  WGS 84
\end{verbatim}

\begin{Shaded}
\begin{Highlighting}[]
\CommentTok{\#Repeat this process for the data with country names and inequality indices }
\CommentTok{\#\textquotesingle{}skip = 5\textquotesingle{} removes the first five rows, which are all header material}
\CommentTok{\#\textquotesingle{}na = ".." removes na values which are stored as ".." in this data set}
\CommentTok{\#This is necessary as we need the values in the index column to be recognized as numeric}
\CommentTok{\#remove\_empty() removes the blank columns in this data set, making it neater}
\CommentTok{\#\textquotesingle{}quiet = True\textquotesingle{} stops the program declaring the columns it\textquotesingle{}s removed}
\CommentTok{\#\textquotesingle{}\#clean\_names also adds an "x" to columns with numeric titles, making later calculations possible}

\NormalTok{InequalityData }\OtherTok{\textless{}{-}} \FunctionTok{read\_csv}\NormalTok{(}\FunctionTok{here}\NormalTok{(}\StringTok{"Gender Inequality Index (GII).csv"}\NormalTok{),}
                           \AttributeTok{skip =} \DecValTok{5}\NormalTok{, }\AttributeTok{na =} \StringTok{".."}\NormalTok{,}
                           \AttributeTok{locale =} \FunctionTok{locale}\NormalTok{(}\AttributeTok{encoding =} \StringTok{"latin1"}\NormalTok{)) }\SpecialCharTok{\%\textgreater{}\%}
  \FunctionTok{remove\_empty}\NormalTok{(}\AttributeTok{which =} \StringTok{"cols"}\NormalTok{, }\AttributeTok{quiet =} \ConstantTok{TRUE}\NormalTok{) }\SpecialCharTok{\%\textgreater{}\%}
  \FunctionTok{clean\_names}\NormalTok{() }\SpecialCharTok{\%\textgreater{}\%}
  \FunctionTok{slice}\NormalTok{(}\DecValTok{1}\SpecialCharTok{:}\DecValTok{189}\NormalTok{) }\SpecialCharTok{\%\textgreater{}\%}
  \CommentTok{\#Creates a new column containing an iso code to match columns in CountryData}
  \FunctionTok{mutate}\NormalTok{(}\AttributeTok{iso\_code=}\FunctionTok{countrycode}\NormalTok{(country, }\AttributeTok{origin =} \StringTok{\textquotesingle{}country.name\textquotesingle{}}\NormalTok{, }\AttributeTok{destination =} \StringTok{\textquotesingle{}iso2c\textquotesingle{}}\NormalTok{))}

\CommentTok{\#The country names from the two dataframes cannot be merged as it stands}
\CommentTok{\#This is because InequaliyData has a blank space " " before the country name}
\CommentTok{\#This block of code fixes this problem}
\CommentTok{\#Because of countrycode, this section now isn\textquotesingle{}t strictly necessary}

\NormalTok{CountryList }\OtherTok{\textless{}{-}}\NormalTok{ dplyr}\SpecialCharTok{::}\FunctionTok{select}\NormalTok{(InequalityData, country) }\CommentTok{\#Extracts countries as a list}
\NormalTok{CountryListTrimmed }\OtherTok{\textless{}{-}} \FunctionTok{as.list}\NormalTok{(}\FunctionTok{trimws}\NormalTok{(CountryList}\SpecialCharTok{$}\NormalTok{country, }\StringTok{"l"}\NormalTok{)) }\CommentTok{\#Removes the blank spaces}
\CommentTok{\#Adds this fixed list back into the dataframe}
\NormalTok{CleanInequalityData }\OtherTok{\textless{}{-}} \FunctionTok{mutate}\NormalTok{(InequalityData, CountryListTrimmed)}

\CommentTok{\#This block merges the two dataframes together using their common country names columns}
\CommentTok{\#**Some data is lost because some country names do not match perfectly}
\CommentTok{\#The code also produces a new column with the change of index from columns "x2010" and "x2019"}

\NormalTok{JoinedDataFrame }\OtherTok{\textless{}{-}} \FunctionTok{merge}\NormalTok{(CountryData, CleanInequalityData,}
                         \CommentTok{\#The titles of the columns with the common value names}
                         \AttributeTok{by.x =} \StringTok{"iso"}\NormalTok{, }\AttributeTok{by.y =} \StringTok{"iso\_code"}\NormalTok{) }\SpecialCharTok{\%\textgreater{}\%}
  \CommentTok{\#Creates the new comparison column}
  \FunctionTok{mutate}\NormalTok{(., }\AttributeTok{InequalityDifference2010s =}\NormalTok{ x2010 }\SpecialCharTok{{-}}\NormalTok{ x2019) }\SpecialCharTok{\%\textgreater{}\%}
  \CommentTok{\#Reduces the number of columns to just the important ones we want}
  \FunctionTok{select}\NormalTok{(country.x, iso, geometry, InequalityDifference2010s)}

\CommentTok{\#This calculates and prints a mean value for the change in inequality index across the world}
\CommentTok{\#"na.rm = TRUE" removes the na values, which prevents "na" from being returned as our output}
\NormalTok{MeanChange }\OtherTok{\textless{}{-}} \FunctionTok{mean}\NormalTok{(JoinedDataFrame}\SpecialCharTok{$}\NormalTok{InequalityDifference2010s, }\AttributeTok{na.rm =} \ConstantTok{TRUE}\NormalTok{)}
\FunctionTok{print}\NormalTok{(MeanChange)}
\end{Highlighting}
\end{Shaded}

\begin{verbatim}
## [1] 0.04991391
\end{verbatim}

\begin{Shaded}
\begin{Highlighting}[]
\CommentTok{\#Plot the map with the values colourising the countries}
\FunctionTok{tm\_shape}\NormalTok{(JoinedDataFrame) }\SpecialCharTok{+}
  \FunctionTok{tm\_polygons}\NormalTok{(}
    \AttributeTok{col =} \StringTok{"InequalityDifference2010s"}\NormalTok{,}
    \AttributeTok{palette=}\StringTok{"RdYlGn"}\NormalTok{, }\CommentTok{\#Red, Yellow, Green Pallette}
    \AttributeTok{style=}\StringTok{"pretty"}\NormalTok{, }\CommentTok{\#Pretty is one of the colouring styles}
    \AttributeTok{n=}\DecValTok{8}\NormalTok{, }\CommentTok{\#Sets eight colour categories}
    \AttributeTok{midpoint =} \FloatTok{0.1}\NormalTok{) }\CommentTok{\#The value for the bland colour between yellow and green}
\end{Highlighting}
\end{Shaded}

\includegraphics{HomeworkCommentary_files/figure-latex/unnamed-chunk-1-1.pdf}

\end{document}
